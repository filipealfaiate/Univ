\documentclass[11pt]{article}

\usepackage[a4paper, total={16cm, 24cm}]{geometry}
\usepackage[portuguese]{babel}
\usepackage[utf8]{inputenc}
\usepackage{graphicx}
\usepackage{amsmath}
\usepackage{tikz}
    \usetikzlibrary{shadows}
\usepackage{booktabs}
\usepackage[colorlinks=true]{hyperref}
\usepackage{listings}
    \renewcommand\lstlistingname{Listagem}
    \lstset{numbers=left, numberstyle=\tiny, numbersep=5pt, basicstyle=\footnotesize\ttfamily, frame=tb,rulesepcolor=\color{gray}, breaklines=true}
\usepackage{blindtext}

% -------------------------------------------------------------------------------------------
\title
{
    \includegraphics[width=0.4\textwidth]{imgs/university.png}
    \\[0.1cm]
    \textbf{4º Trabalho} \\
    Inteligência Artificial
}

\author
{
    \textbf{Professora:} Irene Rodrigues \\
    \textbf{Realizado por:} Filipe Alfaiate (43315), Miguel de Carvalho (43108), João Pereira (42864) 
}
\date{\today}

% -------------------------------------------------------------------------------------------
%                                Body                                                       %
% -------------------------------------------------------------------------------------------

\begin{document}
\maketitle

% -------------------------------------------------------------------------------------------
\section{}

\textbf{Vocabulário}

\hspace{0,4cm}\verb|Condições|:

\hspace{0,8cm}\verb|Eternas|:

\hspace{1,2cm} \verb|linha(L, C, C1)| - A linha \textbf{L} com origem no local \textbf{C} e 
com o destino \textbf{C1}

\hspace{0,8cm} \verb|Fluentes|:

\hspace{1,2cm} \verb|objLocal(O, L)| - O objeto \textbf{O} está no local \textbf{L} 

\hspace{1,2cm} \verb|combLocal(C, L)| - O comboio \textbf{C} está no local \textbf{L} 

\hspace{1,2cm} \verb|objComb(O, C)| - O objeto \textbf{O} está no comboio \textbf{C}


\hspace{0,4cm}\verb|Ações|:

\hspace{0,8cm} \verb|desloca(C, L)| - O comboio \textbf{C} desloca-se para o local \textbf{L}

\hspace{0,8cm} \verb|coloca(O, C)| - O objeto \textbf{O} é colocado dentro do comboio \textbf{C}

\hspace{0,8cm} \verb|tira(O, C)| - O objeto \textbf{O} é retirado de dentro do comboio \textbf{C}


\section{}

\textbf{Estado Inicial:}

\begin{lstlisting}
    estado_inicial([
                objLocal(1, porto),
                objLocal(2, lisboa),
                objLocal(3, lisboa),
                objLocal(4, evora),
                objLocal(5, evora),
                combLocal(1, lisboa),
                combLocal(2, lisboa),
                linha(1, lisboa, porto),
                linha(1, porto, lisboa),
                linha(2, lisboa, evora),
                linha(2, evora, lisboa)
                ]).
\end{lstlisting}

\newpage
\textbf{Estado Final:}

\begin{lstlisting}
    estado_final1([
        objLocal(1, evora),
        objLocal(2, porto),
        objLocal(3, evora),
        objLocal(4, porto),
        objLocal(5, lisboa)
        ]).
\end{lstlisting}

\section{}

\hspace{0,6cm}Considerando o estado inicial da pergunta anterior até ao estado 1 escrito nesta pergunta
a solução do \verb|pop| seria:

\begin{lstlisting}
    [s1-inicial,
     s5-desloca(1, porto),
     s4-coloca(1, 1),
     s427-desloca(1, lisboa),
     s3-tira(1, 1),
     s2-final]
\end{lstlisting}

Um exemplo de \textbf{ameaça}:

\begin{lstlisting}
    ameaca(s5, s4, s427, combLocal(1, porto))
\end{lstlisting}

Significa que o \verb|s427| é ameaçado pelo \verb|s5| e \verb|s4|, o que vai originar uma
ordenação topológica \verb|s5 < s4 < s427|.

Tendo todas as \textbf{ameaças resolvidas} vamos obter uma ordenação topológica do resultado
final \verb|s1 < s5 < s4 < s427 < s3 < s2|.

\section{}

\hspace{0,6cm}A solução deste problema não pode ser obtida diretamente pois o programa excede
o limite de memória. Para solucionar este problema, dividimos em sub-problemas, de forma a não
exceder o limite de memória.

\textbf{1º Sub-problema}:

\hspace{1,2cm}\textbf{Estado Inicial:}

\begin{lstlisting}
    estado_inicial([
                objLocal(1, porto),
                objLocal(2, lisboa),
                objLocal(3, lisboa),
                objLocal(4, evora),
                objLocal(5, evora),
                combLocal(1, lisboa),
                combLocal(2, lisboa),
                linha(1, lisboa, porto),
                linha(1, porto, lisboa),
                linha(2, lisboa, evora),
                linha(2, evora, lisboa)
                ]).
\end{lstlisting}

\newpage
\hspace{1,2cm}\textbf{Estado Final:}

\begin{lstlisting}
    estado_final([
                objLocal(1, lisboa),
                objLocal(2, lisboa),
                objLocal(3, lisboa),
                objLocal(4, lisboa),
                objLocal(5, lisboa)
                ]).
\end{lstlisting}

Considerando os \textbf{estados iniciais e finais} referidos acima, a solução \verb|pop| seria

\begin{lstlisting}
    [s1-inicial,
     s1059-desloca(2, evora),
     s5-desloca(1, porto),
     s1009-coloca(4, 2),
     s1104-coloca(5, 2),
     s4-coloca(1, 1),
     s1083-desloca(2, lisboa),
     s427-desloca(1, lisboa),
     s1096-tira(5, 2),
     s3-tira(1, 1),
     s428-tira(4, 2),
     s2-final]
\end{lstlisting}

Um exemplo de \textbf{ameaça}:

\begin{lstlisting}
    ameaca(s1059, s1104, s1083, combLocal(2, evora))
\end{lstlisting}

Significa que o \verb|s1083| é ameaçado pelo \verb|s1059| e \verb|s1104|, o que vai originar uma
ordenação topológica \verb|s1059 < s1104 < s1083|.

\newpage

\textbf{2º Sub-problema}:

\hspace{1,2cm}\textbf{Estado Inicial:}

\begin{lstlisting}
    estado_inicial([
                objLocal(1, lisboa),
                objLocal(2, lisboa),
                objLocal(3, lisboa),
                objLocal(4, lisboa),
                objLocal(5, lisboa),
                combLocal(1, lisboa),
                combLocal(2, lisboa),
                linha(1, lisboa, porto),
                linha(1, porto, lisboa),
                linha(2, lisboa, evora),
                linha(2, evora, lisboa)
                ]).
\end{lstlisting}

\hspace{1,2cm}\textbf{Estado Final:}

\begin{lstlisting}
    estado_final([
        objLocal(1, evora),
        objLocal(2, porto),
        objLocal(3, evora),
        objLocal(4, porto),
        objLocal(5, lisboa)
        ]).
\end{lstlisting}

Considerando os \textbf{estados iniciais e finais} referidos acima, a solução \verb|pop| seria

\begin{lstlisting}
    [s1-inicial,
     s11938-coloca(1, 2),
     s17552-coloca(2, 1),
     s18110-coloca(3, 2),
     s18153-coloca(4, 1),
     s17129-desloca(2, evora),
     s17967-desloca(1, porto),
     s17551-tira(2, 1),
     s18017-tira(3, 2),
     s18152-tira(4, 1),
     s3-tira(1, 2),
     s2-final]
\end{lstlisting}

Um exemplo de \textbf{ameaça}:

\begin{lstlisting}
    ameaca(s1, s18153, s17967, combLocal(1, lisboa))
\end{lstlisting}

Significa que o \verb|s17967| é ameaçado pelo \verb|s18153| e \verb|s1|, o que vai originar uma
ordenação topológica \verb|s1 < s18153 < s17967|.

\newpage
\textbf{Solução do Problema}:

\hspace{1,2cm}\textbf{Estado Inicial}:

\begin{lstlisting}
    estado_inicial([
                objLocal(1, porto),
                objLocal(2, lisboa),
                objLocal(3, lisboa),
                objLocal(4, evora),
                objLocal(5, evora),
                combLocal(1, lisboa),
                combLocal(2, lisboa),
                linha(1, lisboa, porto),
                linha(1, porto, lisboa),
                linha(2, lisboa, evora),
                linha(2, evora, lisboa)
                ]).
\end{lstlisting}

\hspace{1,2cm}\textbf{Estado Final:}

\begin{lstlisting}
    estado_final([
        objLocal(1, evora),
        objLocal(2, porto),
        objLocal(3, evora),
        objLocal(4, porto),
        objLocal(5, lisboa)
        ]).
\end{lstlisting}

Considerando os \textbf{estados iniciais e finais} referidos acima, a solução \verb|pop| seria

\begin{lstlisting}
    [s1-inicial,
     s1059-desloca(2, evora),
     s5-desloca(1, porto),
     s1009-coloca(4, 2),
     s1104-coloca(5, 2),
     s4-coloca(1, 1),
     s1083-desloca(2, lisboa),
     s427-desloca(1, lisboa),
     s1096-tira(5, 2),
     s6-tira(1, 1),
     s428-tira(4, 2),
     s11938-coloca(1, 2),
     s17552-coloca(2, 1),
     s18110-coloca(3, 2),
     s18153-coloca(4, 1),
     s17129-desloca(2, evora),
     s17967-desloca(1, porto),
     s17551-tira(2, 1),
     s18017-tira(3, 2),
     s18152-tira(4, 1),
     s3-tira(1, 2),
     s2-final]
\end{lstlisting}

Devido ao programa não executar num todo, não foi possível calcular as ameaças, contudo 
nos sub-problemas usados para calcular o \textbf{problema total} estão referidas algumas ameaças

Tendo todas as \textbf{ameaças resolvidas} vamos obter uma ordenação topológica do resultado
final 
\begin{lstlisting}
    s1 < s1059 < s5 < s1009 < s1104 < s4 < s1083 < s427 < s1096 < s6 < s428 < s11938 < s17552 < s18110 < s18153 < s17129 < s17967 < s17551 < s18017 < s18152 < s3 < s2
\end{lstlisting}
% -------------------------------------------------------------------------------------------
\end{document}