\documentclass[11pt]{article}

\usepackage[a4paper, total={16cm, 24cm}]{geometry}
\usepackage[portuguese]{babel}
\usepackage[utf8]{inputenc}
\usepackage{graphicx}
\usepackage{amsmath}
\usepackage{tikz}
    \usetikzlibrary{shadows}
\usepackage{booktabs}
\usepackage[colorlinks=true]{hyperref}
\usepackage{listings}
    \renewcommand\lstlistingname{Listagem}
    \lstset{numbers=left, numberstyle=\tiny, numbersep=5pt, basicstyle=\footnotesize\ttfamily, frame=tb,rulesepcolor=\color{gray}, breaklines=true}
\usepackage{blindtext}

% -------------------------------------------------------------------------------------------
\title
{
    \includegraphics[width=0.4\textwidth]{university.png}
    \\[0.1cm]
    \textbf{1ª Versão} \\
    Sistemas Móveis e Aplicações
}

\author
{
    \textbf{Professor:} Vítor Nogueira \\
    \textbf{Realizado por:} Miguel de Carvalho (43108), Filipe Alfaiate (43315)
}
\date{\today}

% -------------------------------------------------------------------------------------------
%                                Body                                                       %
% -------------------------------------------------------------------------------------------

\begin{document}
\maketitle

% -------------------------------------------------------------------------------------------
\begin{itemize}
    \item Nome da aplicação: \textbf{Virtual Tripmeter}
    \item Objetivos: 
    \begin{itemize}
        \item Geral:
        \begin{itemize}
            \item Ajudar os Co-Pilotos de \textbf{Rally} a conseguirem treinar a leitura dos \textbf{RoadBooks} (indicações do caminho), simulando o funcionamento do equipamento \textbf{TerraTrip}. É utilizado para registar as distâncias percorridas (Parciais e Totais), o que auxilia na leitura do \textbf{RoadBook}, pois as \textbf{notas} apresentam sempre uma \textbf{distância parcial} entre elas e apresenta também a \textbf{distância total} desde o início até à posição atual do carro.
        \end{itemize}
        \item Especifico:
        \begin{itemize}
            \item A aplicação suporta a língua Inglesa e Portuguesa;
            \item Basta o utilizador criar ou entrar na aplicação utilizando a sua conta;
            \item As definições do utilizador são sincronizadas através da Cloud;
            \item Simulação da estrutura de um \textbf{TerraTrip}, ou seja, 2 contadores de unidade de distância calculados pela "velocidade" a que o veículo se desloca. As caixas podem sofrer um \textit{reset} com um toque longo ou com um simples toque;
            \item Possibilidade de aumentar ou diminuir a velocidade atual ao tocar nos \textbf{botões} "+" e "-", caso seja feito um toque longo, o aumento de velocidade será maior;
            \item Botões que permitem \textbf{começar} (play), \textbf{parar} (pause) e \textbf{terminar} (stop) a simulação do \textbf{TerraTrip};
            \item Possibilidade de usar \textbf{unidades imperiais} (Milhas) em vez de \textbf{unidades métricas} (Quilómetros);
            \item Escolher uma velocidade base;
            \item Possibilidade de emitir um alerta quando é realizado um \textit{reset} num dos contadores.
        \end{itemize}
    \end{itemize}
    \item Dificuldades:
    \begin{itemize}
        \item Entender o funcionamento da linguagem \textit{Kotlin} e as suas diferenças face a uma linguagem já conhecida por nós, o \textit{Java};
        \item Entender a estrutura e as diferentes componentes que constituem uma aplicação \textit{Android};
        \item Perceber o funcionamento da plataforma \textit{FireBase} e perceber a facilidade com que permite gerir utilizadores, usar bases de dados, gerar estatísticas e guardar ficheiros.
    \end{itemize}
\end{itemize}

% -------------------------------------------------------------------------------------------
\end{document}